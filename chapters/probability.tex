%!TEX root = ../lecture_notes.tex


\section{Probability}
\label{cap:probability}

\subsection{Introduction}
\label{cap:prob_intro}

\textbf{NB:} in this chapter, we follow \cite{pml1Book}.\\

\noindent The field of probability studies, from a quantitative perspective, how \emph{likely} an event is. Conceptually, and perhaps historically, there are two main interpretations of probability. The first one is \textbf{frequentist probability}, which relates to frequency of occurrence, and then applies only to event that can be repeated an infinite number of times, such as throwing a dice or flopping a coin. As a consequence, this standpoint fails to assign a probability to events that are impossible to repeat, such as the average temperature of the Earth's surface reaching an all-time maximum in the year 2025. A second interpretation is that of \textbf{Bayesian probability}, which represents uncertainty about the occurrence of an event. This uncertainty might come from different sources, such unknown features in the experiments (epistemological uncertainty) or random components (aleatoric uncertainty). In this case, events need not be repeatable be be assigned with a probability.

A basic knowledge of probability theory, definitions and results in fundamental in ML. This is because in ML we design, train and deploy mathematical models that i) aim to capture/quantify uncertainty, and ii) deal with noise-corrupted training data. Therefore, a rigorous account of uncertainty is central to real-world ML applications.


\subsubsection{Definitions}
\label{cap:prob_defs}

To start studying probability, we will focus on the outcome $\omega$ of a hypothetical experiment, e.g., throwing a dice, where $\omega$ can take values $\{1,2,3,4,5,6\}$. In this context, we can define: 

\begin{definition}[Sample space] The set containing all the possible outcomes $\omega$ of an experiment is called sample space and is denoted by $\Omega$.
\end{definition}

\begin{definition}[Event space] The set $\cA$, referred to as event space, contains all possible subset of the sample space $\Omega$. Therefore, each element $A\in\cA$ represents a possible results of the experiment.
\end{definition}

\begin{definition}[Probability] The function
\begin{align}
 	 \Pb: \cA &\to [0,1]\\
 	 x &\mapsto \Pb(x)
 \end{align}
  denotes the probability of the result of the experiment falling inside the elements of $\cA$, that is, $\Pb(A) = \Pb(\omega\in A)$. The function $\Pb$ needs to fulfil some standard properties such as $\Pb(\Omega) = 1, \Pb(\emptyset) = 0,$ and $\Pb(A^c) = 1-\Pb(A)$.
\end{definition}

\begin{mdframed}[style=discusion, frametitle={\center $\cA$ vs $\Omega$}]

Why is the probability defined over $\cA$ and not over $\Omega$? Discuss via some examples

\end{mdframed}

\begin{mdframed}[style=ejemplo, frametitle={\center Examples}]
\felipe{Consider basic examples (dice, coin, uniform, rain), and present the sample space, event space, and probability}
\end{mdframed}

We refer to the triplet $(\Omega,\cA,\Pb)$ as \textbf{probability space}. 

\subsubsection{Basic properties}
\label{cap:prob_basicprop}

The joint probability of events $A$ and $B$ is denoted by
\begin{equation}
	\Pb(A\wedge B) =\Pb(A\cap B) = \Pb(A,B),
\end{equation}
note that if $\omega\in A$ and $\omega\in B$, then, $\omega\in A\cap B$.

The conditional probability of the event $A$ occurring, given that the event $B$ occurred is denoted by 
\begin{equation}
	\Pb(A|B) = \frac{\Pb(A,B)}{\Pb(B)},
\end{equation}
which is only valid when $\Pb(B)>0$.

Additionally, we say that events $A$ and $B$ are \textbf{independent} iff $\Pb(A,B)=\Pb(A)\Pb(B)$. Observe that this implies that 
\begin{equation}
	\Pb(A|B) = \frac{\Pb(A,B)}{\Pb(B)}= \frac{\Pb(A)\Pb(B)}{\Pb(B)} = \Pb(A),
\end{equation}
meaning that when $A$ and $B$ are independent, the latter provides no \textbf{information} for $A$.

Lastly, the probability of intersection, i.e., the probability of the events $\omega\in A$ or $\omega\in B$ is given by 
\begin{equation}
	\Pb(A \vee B) = \Pb(A) + \Pb(B) - \Pb(A \wedge B).
\end{equation}

\subsection{Random variables}
\label{cap:prob_RV}




